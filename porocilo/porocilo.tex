\documentclass[11pt,a4paper]{article}

\usepackage[slovene]{babel}
\usepackage[utf8x]{inputenc}
\usepackage{graphicx}

\pagestyle{plain}

\begin{document}
\title{Poročilo pri predmetu \\
Analiza podatkov s programom R}
\author{Katarina Černe}
\maketitle

\section{Izbira teme}

Pri predmetu ANPP sem si za temo projekta izbrala Obsojene fizične osebe po občinah stalnega prebivališča, Slovenija.

Analizirala bom, kje v Sloveniji je največ obsojenih, v katerih občinah procent obsojenih najbolj narašča in kje najbolj pada od leta 2006 do leta 2013. Iskala bom, če so kakšne povezave med obsojenimi ter brezposelnostjo in med diplomanti.

Cilj projekta je skozi analizo podatkov spoznati program R.

\section{Obdelava, uvoz in čiščenje podatkov}

Obdelavo podatkov sem pričela z uvozom cvs datotek iz spletne strani Statističnega urada.
Naredila sem dve tabeli in sicer, prva je o obsojenih fizičnih osebah po regijah v Sloveniji od leta 2006 do leta 2013 - datoteka obsojenir.csv, druga pa o diplomantih terciarnega izobraževanja po statistični regiji stalnega prebivališča, prav tako od leta 2006 do leta 2013 - datoteka diplomantir.csv.
Nato sem v datoteki uvoz napisala funkciji za obe tabeli - uvoz1.r je tabela o obsojenih fizičnih osebah, uvoz2.r pa tabela o diplomantih.
Narisala sem tudi graf o obsojenih fizičnih osebah v letu 2007 po regijah - pitao.R in graf o diplomantih po regijah v letu 2007 - grafd.R in sliki shranila v pdf obliki v mapo slike.
Pri grafu o diplomantih imam problem, ker mi na x osi namesto imen regij izpiše zaporedno številko.

\includegraphics[width=\textwidth]{../slike/leto_2007_obsojeni.pdf}
\includegraphics[width=\textwidth]{../slike/leto_2007_diplomanti.pdf}

\section{Analiza in vizualizacija podatkov}

%\includegraphics{../slike/povprecna_druzina.pdf}

\section{Napredna analiza podatkov}

%\includegraphics{../slike/naselja.pdf}

\end{document}
