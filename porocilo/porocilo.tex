\documentclass[11pt,a4paper]{article}

\usepackage[slovene]{babel}
\usepackage[utf8x]{inputenc}
\usepackage{graphicx}

\pagestyle{plain}

\begin{document}
\title{Poročilo pri predmetu \\
Analiza podatkov s programom R}
\author{Katarina Černe}
\maketitle

\section{Izbira teme}

Pri predmetu ANPP sem si za temo projekta izbrala Obsojene fizične osebe po občinah stalnega prebivališča, Slovenija.

Analizirala bom, kje v Sloveniji je največ obsojenih, v katerih občinah procent obsojenih najbolj narašča in kje najbolj pada od leta 2006 do leta 2013. Iskala bom, če so kakšne povezave med obsojenimi ter brezposelnostjo in med diplomanti.

Viri podatkov:

*Obsojene fizične osebe: http://pxweb.stat.si/pxweb/Dialog/varval.asp?ma=1372201s&ti=&path=../Database/Dem_soc/13_kriminaliteta/01_statistika_toz_sodisc/10_13722_obsojene_kazalniki/&lang=2

*Diplomanti: http://pxweb.stat.si/pxweb/Dialog/varval.asp?ma=0955405S&ti=Diplomanti+terciarnega+izobra%9Eevanja+po+ob%E8ini+stalnega+prebivali%9A%E8a%2C+Slovenija&path=../Database/Dem_soc/09_izobrazevanje/08_terciarno_izobraz/02_09554_diplomanti_splosno/&lang=2

*Delovno aktivno prebivalstvo in brezposelnost: http://pxweb.stat.si/pxweb/Dialog/varval.asp?ma=0700960S&ti=&path=../Database/Dem_soc/07_trg_dela/05_akt_preb_po_regis_virih/01_07009_aktivno_preb_mesecno/&lang=2

Cilj projekta je skozi analizo podatkov spoznati program R.

\section{Obdelava, uvoz in čiščenje podatkov}

\section{Analiza in vizualizacija podatkov}

\includegraphics{../slike/povprecna_druzina.pdf}

\section{Napredna analiza podatkov}

\includegraphics{../slike/naselja.pdf}

\end{document}
